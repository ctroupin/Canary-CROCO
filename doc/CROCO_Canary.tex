\documentclass[final,xcolor=table,svgnames]{beamer}
\usepackage[english]{babel}
%\usepackage[latin1]{inputenc}
\usepackage[T1]{fontenc}
\usepackage{graphicx}
\usepackage{hyperref}
\usepackage{url}
\usepackage{tikz}
\usepackage{doi}
\usepackage{fontawesome}
\usepackage{academicons}
\usepackage{fontspec}
\usepackage{ifsym}
\usepackage{tcolorbox}

\setsansfont{calibri}[
Path = /home/ctroupin/.fonts/,
Extension = .ttf ,
BoldFont = *Bold,
ItalicFont = *Italic]

\setmainfont{calibri}[
Path = /home/ctroupin/.fonts/,
Extension = .ttf ,
BoldFont = *Bold,
ItalicFont = *Italic]

\DeclareGraphicsExtensions{.pdf,.png,.PNG,.JPG,.jpg,.jpeg,.gif}


\setbeamertemplate{navigation symbols}{}

\tikzstyle{na} = [baseline=-.5ex]

\parskip .5cm

\definecolor{mycolor1}{HTML}{399ba3}
\definecolor{mycolor2}{HTML}{9addbd}
\definecolor{bluegher}{HTML}{4E519F}
\definecolor{mygrey}{rgb}{0.25,.25,.25}
\definecolor{commentcolor}{rgb}{0.75,.75,.75}
\definecolor{arrowcolor}{rgb}{0.1,0.1,0.1}
\definecolor{alertbg}{HTML}{FEFFBA}
\definecolor{thegreen}{HTML}{52aaab}

\setbeamercolor{title}{fg=black}
\setbeamercolor{institute}{fg=mygrey}
\setbeamercolor{frametitle}{fg=black}
\setbeamercolor{structure}{fg=thegreen}
\setbeamercolor{item projected}{fg=mycolor1}
\setbeamercolor{alerted text}{fg=thegreen,bg=alertbg}

\setbeamercovered{invisible}


\addtobeamertemplate{navigation symbols}{}{%
    \usebeamerfont{footline}%
    \usebeamercolor[fg]{footline}%
    \hspace{1em}%
    \insertframenumber/\inserttotalframenumber
}

\setbeamercolor{footline}{fg=black}
\setbeamerfont{footline}{series=\bfseries}

\setbeamerfont{sectiontitle1}{size=\huge,family=\rmfamily}
\setbeamerfont{sectiontitle2}{size=\fontsize{40}{20}\selectfont,shape=\itshape}
\setbeamerfont{author}{size=\large}
\setbeamerfont{institute}{size=\normalsize,family=\rmfamily}
\setbeamerfont{title}{size=\fontsize{25}{20}\selectfont,shape=\itshape}
\setbeamerfont{subtitle}{size=\normalsize,family=\rmfamily}
\setbeamerfont{projected text}{size=\normalsize}
%\setbeamertemplate{enumerate items}[$\tilde{}$]
\setbeamertemplate{itemize item}{\footnotesize$\boldsymbol{\sim}$}
\setbeamersize{description width=.5cm}
%\setbeamersize{text margin left=1em}

\setbeamertemplate{enumerate items}[circle]
\setbeamercolor{item projected}{bg=mycolor1,fg=white}

%--------------------------------
\hypersetup{bookmarksopen=true,
bookmarksnumbered=true,  
pdffitwindow=true, 
pdfstartview=Fit,
%pdfpagemode=FullScreen,
pdffitwindow=true,
pdftoolbar=true,
pdfmenubar=true,
pdfwindowui=true,
pdfsubject={CROCO, modelling},
pdfauthor={C. Troupin},
bookmarksopenlevel=0,
colorlinks=true,
linkcolor=bluegher,anchorcolor=black,%
citecolor=bluegher,filecolor=black,%
menucolor=black,urlcolor=bluegher,%
pdfpageduration=1,%
}

\newcommand{\listcheck}{\textcolor{DarkGreen}{\checkmark}}
%\newcommand{\listcross}{\textcolor{red}{$\times$}}

\newcommand{\mycomment}[1]{\hfill\textcolor{LightGray}{(#1)}}

\setbeamertemplate{frametitle}[default][left,leftskip=1cm] % <-- choose here the leftskip you wish
\newcommand{\mydoi}[1]{\textcolor{mycolor1}{\aiDoi~\href{https://doi.org/#1}{#1}}}


\title{Hydrodynamic modelling of the Canary Islands}
\subtitle{}
\author[C.~Troupin]{Charles Troupin}
%\institute{GHER, ULiège}
\date{\today}
%\titlegraphic{\includegraphics[height=\logoheight]{logo_uliege_square}\includegraphics[height=\logoheight]{logo_gher}}
  
\urlstyle{rm}
\begin{document}
{
%\usebackgroundtemplate{\tikz\node[opacity=0.5] {\hspace{-1cm}\includegraphics[width=1.2\paperwidth]{Athens_2155}}; }
\begin{frame}
\maketitle
\end{frame}
}
%-----------------------------------------------------------------------------------------------

\begin{frame}[c]
\frametitle{Initial motivations}

\begin{enumerate}[label=\protect\mycirc{\color{white}\arabic*}]
\item Have a working implementation of CROCO around the islands
\item Simulate past situations using a realistic configuration
\item Perform process-oriented numerical experiments
\end{enumerate}

\end{frame}


%-----------------------------------------------------------------------------------------------

\begin{frame}{Sea surface temperature in September 2022}
\begin{figure}[h!]
\centering    
\movie[width=1.0\textwidth,autostart,showcontrols,loop] 
  {\includegraphics[width=0.8\textwidth]{SST_20220901_030120_nest.png}}{croco_anim.mp4}
\end{figure} 
\end{frame}
 
%-----------------------------------------------------------------------------------------------

\begin{frame}[c]
\frametitle{The model: CROCO}

\textbf{C}oastal and \textbf{R}egional \textbf{O}cean \textbf{CO}mmunity model

\url{https://www.croco-ocean.org/}

\begin{itemize}
\item Simulation of high-resolution dynamics in the coastal ocean
\item Coupling with atmosphere, sediments, biogeochemistry
\item Non-hydrostatic capabilities
\item Large user community  \hfill \url{https://forum.croco-ocean.org/}
\end{itemize}

\end{frame}

%-----------------------------------------------------------------------------------------------

\begin{frame}[c]
\frametitle{Pre-processing: creation of input files}
\begin{enumerate}[label=\protect\mycirc{\color{white}\arabic*}]
\item Grid(s) \mycomment{if nesting}
\item Initial conditions
\item Boundary conditions
\item Atmospheric forcing
\end{enumerate}

\end{frame}

%-----------------------------------------------------------------------------------------------

\begin{frame}[c]

\frametitle{Pre-processing: 2 main toolboxes available}

\begin{description}
\item[CROCO\_PYTOOLS:] scripts in Python for the creation of the input files (except the forcing!)\\
\url{https://croco-ocean.gitlabpages.inria.fr/croco_pytools/}
\item[ROMS-Tools] script in Python to work with UCLA-ROMS\\
\url{https://roms-tools.readthedocs.io/en/latest/}
\end{description}


\end{frame}

%-----------------------------------------------------------------------------------------------

\begin{frame}[t]
\frametitle{Main grid}

Parent grid: inspired from E.~Mason (2011)% \hfill \mydoi{10.1029/2010JC006665}
Child grid: Canary islands and Cape Ghir

\begin{columns}
\begin{column}{0.7\textwidth}
\begin{figure}
\centering
\includegraphics[width=.99\textwidth]{croco_domains.png}
\end{figure}
\end{column}
\begin{column}{0.3\textwidth}
%Script: \texttt{make\_grid.py}\\
Improvement: Strait of Gibraltar
\end{column}
\end{columns}

\end{frame}

%-----------------------------------------------------------------------------------------------

\begin{frame}[c]
\frametitle{Grids and spatial resolution}
\begin{description}
\item[Global grid:] GLORYS12V1 -- 1/12°% \hfill \mydoi{10.48670/moi-00021}
\item[Parent grid:] $\approx$ 3.75~km
\item[Child grid:] $\approx$ 1~km
\item[Child grid \#2] $\approx$ 300~m \mycomment{not implemented yet}
\end{description}
\textbf{Note:} nesting factor should be an odd number\\
(typically 3 or 5)
\end{frame}

%-----------------------------------------------------------------------------------------------

\begin{frame}[c]
\frametitle{Atmospheric forcing}
Two main approaches:
\begin{enumerate}[label=\protect\mycirc{\color{white}\arabic*}]
\item \textbf{Bulk} fluxes: use atmospheric state variables and compute fluxes
\item \textbf{Net} fluxes: use pre-calculated atmospheric forcing fields
\end{enumerate}

\begin{description}
\item[CROCO\_PYTOOLS:] bulk method only
\item[ROMS-Tools:] code is available, but the data download takes hours 
\end{description}

\end{frame}


\begin{frame}[c]
\frametitle{Atmospheric forcing: illustration}

\begin{figure}
\centering
\includegraphics<1>[width=.8\textwidth]{forcing_2m_temperature}
\includegraphics<2>[width=.8\textwidth]{forcing_sea_surface_temperature}
\includegraphics<3>[width=.8\textwidth]{forcing_10m_u_component_of_wind}
\includegraphics<4>[width=.8\textwidth]{forcing_10m_V_component_of_wind}
\includegraphics<5>[width=.8\textwidth]{forcing_specific_humidity}
\includegraphics<6>[width=.8\textwidth]{forcing_surface_net_solar_radiation}
\includegraphics<7>[width=.8\textwidth]{forcing_total_precipitation}
\includegraphics<8>[width=.8\textwidth]{forcing_windvec}

\end{figure}

\end{frame}

\begin{frame}[t]
\frametitle{Initial conditions: extracted from GLORYS12V1}

\onslide<1-2>{Parent grid}
\onslide<3-4>{Child grid}
\begin{figure}
\centering
\includegraphics<1>[width=.8\textwidth]{croco_ini_nea_hermione_mercator_Y2022M09_temp.png}
\includegraphics<2>[width=.8\textwidth]{croco_ini_nea_hermione_mercator_Y2022M09_salt.png}
\includegraphics<3>[width=.8\textwidth]{croco_ini_nea_mercator_Y2022M09_temp.png}
\includegraphics<4>[width=.8\textwidth]{croco_ini_nea_mercator_Y2022M09_salt.png}
\end{figure}
\onslide*<5>{GLORYS already rich in small-scale features}
\end{frame}



\begin{frame}[c]
\frametitle{Nesting with AGRIF}

AGRIF = \textbf{A}daptative \textbf{G}rid \textbf{R}efinement \textbf{i}n \textbf{F}ortran\\
Debreu et al. (2008) %\hfill \mydoi{10.1016/j.cageo.2007.01.009}\\
\url{https://agrif.imag.fr}

\begin{itemize}
\item multi-level nesting in numerical models %\comment{not only ROMS}
\item allows two-way interaction
\end{itemize}

Pre-processing: same steps as parent grid except that the boundary conditions don't have to be extracted from GLORYS

\end{frame}

\begin{frame}[c]
\frametitle{1-way vs 2-way nesting}

% 2-way: the child (finer) domain feedbacks to the parent domain

\begin{figure}
\centering
\includegraphics[width=.45\textwidth]{SST_20220923_090400.png}~\includegraphics[width=.45\textwidth]{SST_20220923_090120_nest.png}\\
\includegraphics[width=.45\textwidth]{SST_20220923_090400_2way.png}~\includegraphics[width=.45\textwidth]{SST_20220923_090120_nest_2way.png}\\
\end{figure}


\end{frame}



\begin{frame}
\frametitle{Computing \& Machine configuration}

Cluster \texttt{NIC5} from the \textit{Consortium des Équipements de Calcul Intensif} (C\'{E}CI -- \url{https://www.ceci-hpc.be/clusters.html})

\begin{itemize}
% \item Hosted at ULiège
\item 4672 CPUs available
\item 2 days max. time for a job
\item Jobs managed through \texttt{Slurm} workload manager \hfill \url{https://slurm.schedmd.com/}
\item Suitable for MPI Parallel jobs
\item Compilation with Intel compilers
\end{itemize}

\end{frame}

%-----------------------------------------------------------------------------------------------

\begin{frame}[c]
\frametitle{Scientific objectives}
\centering
\huge

\begin{enumerate}[label=\protect\mycirc{\color{white}\arabic*}]
\item Process studies \comment{eddies, filaments, wake, \ldots}
\item Setting of an operational (i.e. forecast) configuration
\item Analysis of extreme events \comment{Delta or Hermine storm, \ldots) 
\end{enumerate}
\end{frame}

\end{document}



